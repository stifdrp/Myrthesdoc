\documentclass[]{book}
\usepackage{lmodern}
\usepackage{amssymb,amsmath}
\usepackage{ifxetex,ifluatex}
\usepackage{fixltx2e} % provides \textsubscript
\ifnum 0\ifxetex 1\fi\ifluatex 1\fi=0 % if pdftex
  \usepackage[T1]{fontenc}
  \usepackage[utf8]{inputenc}
\else % if luatex or xelatex
  \ifxetex
    \usepackage{mathspec}
  \else
    \usepackage{fontspec}
  \fi
  \defaultfontfeatures{Ligatures=TeX,Scale=MatchLowercase}
\fi
% use upquote if available, for straight quotes in verbatim environments
\IfFileExists{upquote.sty}{\usepackage{upquote}}{}
% use microtype if available
\IfFileExists{microtype.sty}{%
\usepackage{microtype}
\UseMicrotypeSet[protrusion]{basicmath} % disable protrusion for tt fonts
}{}
\usepackage[margin=1in]{geometry}
\usepackage{hyperref}
\hypersetup{unicode=true,
            pdftitle={Laboratório de jurimetria: documentação},
            pdfauthor={Faculdade de Direito},
            pdfborder={0 0 0},
            breaklinks=true}
\urlstyle{same}  % don't use monospace font for urls
\usepackage{natbib}
\bibliographystyle{apalike}
\usepackage{longtable,booktabs}
\usepackage{graphicx,grffile}
\makeatletter
\def\maxwidth{\ifdim\Gin@nat@width>\linewidth\linewidth\else\Gin@nat@width\fi}
\def\maxheight{\ifdim\Gin@nat@height>\textheight\textheight\else\Gin@nat@height\fi}
\makeatother
% Scale images if necessary, so that they will not overflow the page
% margins by default, and it is still possible to overwrite the defaults
% using explicit options in \includegraphics[width, height, ...]{}
\setkeys{Gin}{width=\maxwidth,height=\maxheight,keepaspectratio}
\IfFileExists{parskip.sty}{%
\usepackage{parskip}
}{% else
\setlength{\parindent}{0pt}
\setlength{\parskip}{6pt plus 2pt minus 1pt}
}
\setlength{\emergencystretch}{3em}  % prevent overfull lines
\providecommand{\tightlist}{%
  \setlength{\itemsep}{0pt}\setlength{\parskip}{0pt}}
\setcounter{secnumdepth}{5}
% Redefines (sub)paragraphs to behave more like sections
\ifx\paragraph\undefined\else
\let\oldparagraph\paragraph
\renewcommand{\paragraph}[1]{\oldparagraph{#1}\mbox{}}
\fi
\ifx\subparagraph\undefined\else
\let\oldsubparagraph\subparagraph
\renewcommand{\subparagraph}[1]{\oldsubparagraph{#1}\mbox{}}
\fi

%%% Use protect on footnotes to avoid problems with footnotes in titles
\let\rmarkdownfootnote\footnote%
\def\footnote{\protect\rmarkdownfootnote}

%%% Change title format to be more compact
\usepackage{titling}

% Create subtitle command for use in maketitle
\newcommand{\subtitle}[1]{
  \posttitle{
    \begin{center}\large#1\end{center}
    }
}

\setlength{\droptitle}{-2em}

  \title{Laboratório de jurimetria: documentação}
    \pretitle{\vspace{\droptitle}\centering\huge}
  \posttitle{\par}
    \author{Faculdade de Direito}
    \preauthor{\centering\large\emph}
  \postauthor{\par}
      \predate{\centering\large\emph}
  \postdate{\par}
    \date{2019-04-02}

\usepackage{booktabs}

\begin{document}
\maketitle

{
\setcounter{tocdepth}{1}
\tableofcontents
}
\hypertarget{visao-geral}{%
\chapter{Visão geral}\label{visao-geral}}

Este documento contêm tudo que se refere ao Myrthes. Nele há seções úteis tanto para desenvolvedores quanto para usuários (estudantes e professores de direito).

\hypertarget{intro}{%
\chapter{Introdução}\label{intro}}

Diante do volume de processos judiciais em versão eletrônica atualmente disponíveis nas páginas dos tribunais, o acesso a esses dados tornou-se possível. Atualmente se vislumbra a possibilidade de baixar processos inteiros e analizá-los. Essa possibilidade, porém, implica em desafios.

A comunidade do direito não é muito afeita à computação, bem como, aos métodos estatísticos e de aprendizado de máquina que atualmente avançam rapidamente e são liderados principalmente por pessoas provenientes da ciência da computação. Acontece que, embora o pessoal da ciência da computação e da estatística tenha liderado a implementação dos métodos de automação de coleta, extração, organização e análise de processos judiciais, ela carece de conhecimento substantivo do direito e isso tem causado erros e equívocos nas coletas e análises de dados.

A necessidade de se implementar um laboratório de jurimetria na faculdade de direito que contemple a interdisciplinariedade se impõe, visto que um laboratório como este exige conhecimentos em várias áreas:

1 - Implementação e manutenção de sistemas distribuídos para lidar com o algo volume de dados;

2 - Conhecimentos específicos sobre processamento de linguagem natural;

3 - Conhecimentos específicos sobre estatística e machine larnings;

4 - Conhecimento aprofundado de cada uma da áreas do direito aplicado diariamente no tribunais de justiça do Brasil.

\hypertarget{req}{%
\chapter{Requerimentos}\label{req}}

\hypertarget{hardware}{%
\section{Hardware}\label{hardware}}

Processador : Mínimo 1GHZ, recomenda-se 2+GHz Quad Core
RAM (exclusive SWAP) :
Com ELK: Min 16 GB, Preferível 32 GB

\hypertarget{network}{%
\section{Network}\label{network}}

As portas 443, 8080, 5432, 9200, 5601, 8983 devem estar abertas. Com exceção da 443, as demais podem estar fechadas, desde que por trás de proxy reverso.

\hypertarget{software}{%
\section{Software}\label{software}}

Ubuntu 18.04+

JAVA 8 JRE ou JDK com JAVA\_HOME configurada no /etc/environment

Python minimal instalado

Nodejs (será usado com o prometheus e grafana)

Locale configurado para pt\_BR.UTF-8

Exigências:

Apache ManifoldCF
Apache Tika
Apache Tomcat
Apache Zookeeper
Apache Solr
ELK (Elasticsearch, Logstash e Kibana)
Prometheus + Grafana
Apache Cassandra
Postgresql 11

\hypertarget{instalacao}{%
\chapter{Instalação}\label{instalacao}}

\hypertarget{arquitetura}{%
\section{Arquitetura}\label{arquitetura}}

\begin{verbatim}
## PhantomJS not found. You can install it with webshot::install_phantomjs(). If it is installed, please make sure the phantomjs executable can be found via the PATH variable.
\end{verbatim}

\hypertarget{implementacao}{%
\section{Implementação}\label{implementacao}}

A tabela abaixo, que deve ser atualizada conforme adicionamos novos elementos, lista as implementações, bem como o seu estágio.

\begin{table}[t]

\caption{\label{tab:tab-imp}Implementações}
\centering
\begin{tabular}{ll}
\toprule
Recurso & Implementado\\
\midrule
scraper tribunais & parcial\\
crawler djes & sim\\
indexação para Solr & sim\\
tika-ocr & sim\\
stopwords & parcial\\
\addlinespace
anotação & não\\
fuzzy search & sim\\
regex textos & sim\\
parseamento htmls & parcial\\
\bottomrule
\end{tabular}
\end{table}

\hypertarget{section}{%
\section{}\label{section}}

\hypertarget{sysconf}{%
\chapter{Configuração do sistema}\label{sysconf}}

\hypertarget{section-1}{%
\section{}\label{section-1}}

\hypertarget{myrconf}{%
\chapter{Configuracao do Myrthes}\label{myrconf}}

Na configuração do sistema é importante levar em consideração o consumo da RAM e da CPU. A indexação consome bastante CPU. Por sua vez, a busca consome muita RAM. Abaixo seguem as configurações mínimas de RAM, de CPU e swap.

\hypertarget{busca}{%
\chapter{Busca com Myrthes}\label{busca}}

\hypertarget{busca-basica}{%
\section{Busca básica}\label{busca-basica}}

\hypertarget{autocompletar}{%
\section{Autocompletar}\label{autocompletar}}

\hypertarget{filtragem-por-datas}{%
\section{Filtragem por datas}\label{filtragem-por-datas}}

\hypertarget{facetas}{%
\section{Facetas}\label{facetas}}

\hypertarget{destaque-highligting}{%
\section{Destaque (highligting)}\label{destaque-highligting}}

\hypertarget{busca-avancada}{%
\section{Busca avançada}\label{busca-avancada}}

\hypertarget{alertas}{%
\section{Alertas}\label{alertas}}

\hypertarget{favoritos}{%
\section{Favoritos}\label{favoritos}}

\hypertarget{configuracao-do-solr-no-myrthes}{%
\section{Configuração do Solr no Myrthes}\label{configuracao-do-solr-no-myrthes}}

\hypertarget{dashbord}{%
\chapter{Dashboard}\label{dashbord}}

\hypertarget{criando-dashboards}{%
\section{Criando dashboards}\label{criando-dashboards}}

\hypertarget{tabelas}{%
\section{Tabelas}\label{tabelas}}

\hypertarget{graficos-de-barra}{%
\section{Gráficos de barra}\label{graficos-de-barra}}

\hypertarget{graficos-de-linha}{%
\section{Gráficos de linha}\label{graficos-de-linha}}

\hypertarget{mapas}{%
\section{Mapas}\label{mapas}}

\hypertarget{textos}{%
\section{Textos}\label{textos}}

\hypertarget{incluindo-facetas-e-filtros-no-dashboard}{%
\section{Incluindo facetas e filtros no dashboard}\label{incluindo-facetas-e-filtros-no-dashboard}}

\hypertarget{partilhando-o-dashboard}{%
\section{Partilhando o dashboard}\label{partilhando-o-dashboard}}

\hypertarget{faq}{%
\chapter{FAQ}\label{faq}}

\bibliography{book.bib}


\end{document}
